We use without proof the fact that topological manifolds are $T_4$.

Therefore for $A, B$ disjoint closed sets there exist disjoint neighborhoods $U_A \supseteq A, U_B \supseteq B$.

By Proposition 2.2.5 there exist smooth bump fuctinos $\psi_A, \psi_B:M\mapsto\bbR$ s.t. 

\begin{enumerate}
\item $\phi_A^{-1}(1) = A$
\item $\phi_B^{-1}(1) = B$
\item $\textrm{supp} \phi_A \subseteq U_A$
\item $\textrm{supp} \phi_B \subseteq U_B$
\end{enumerate}

Note that in  particular this implies

\begin{enumerate}
\item $\phi_A(p) = 0$ for all $p \in B$
\item $\phi_B(p) = 0$ for all $p \in A$
\end{enumerate}


Let $f = \frac{1}{2}\left(1 - \phi_A + \phi_B\right)$.  $f$ is smooth because it is a linear combination of smooth functions, and clearly $0 \leq f(x) \leq 1$ for all $x \in M$.

Then for all $x \in A$

\begin{align*}
f(x) 
	= & \,
\frac{1}{2}\left(1 - \phi_A(x) + \phi_B(x)\right)
	\\
	= & \, 
\frac{1}{2}\left(1 -  1+ 0\right)
	\\
	= & \, 
0.
\end{align*}

Similarly, for all $x \in B$

\begin{align*}
f(x) 
	= & \,
\frac{1}{2}\left(1 - \phi_A(x) + \phi_B(x)\right)
	\\
	= & \, 
\frac{1}{2}\left(1 -  0+ 1\right)
	\\
	= & \, 
1.
\end{align*}
