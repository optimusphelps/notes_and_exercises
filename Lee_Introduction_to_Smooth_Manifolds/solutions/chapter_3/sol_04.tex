First, it is clear that $T\mathbb{S}^1 = \mathbb{S}^1 \times \mathbb{R}$ as a set.  Will attempt to brute force this for learning porpoises.

Pick two points $a, b$ in $\mathbb{S}^1$ and two coordinate charts $(U_{\alpha}, \phi_{\alpha}), (U_{\beta}, \phi_{\beta})$ where $U_\alpha = \mathbb{S}^1 \setminus b, U_{\beta} = \mathbb{S}^1 \setminus a, \phi_{\alpha}(U_{\alpha}) = [-\pi,\pi), \phi_{\beta}(U_{\beta}) = [0,2\pi)$.  It is clear that this is possible by taking the normal identification of $\mathbb{S}^1$ with the unit circle and letting $a = (1,0), b = (-1,0)$.  In addition, note that this choice satisfies the smooth manifold chart lemma.

Now take the standard $(\pi^{-1}(U_{\alpha}), \tilde{\phi_{\alpha}})$ coordinate maps.  Then $\tilde{\phi}_\alpha \circ \tilde{\phi}_\beta^{-1} (x, v) = $

TODO: review and complete